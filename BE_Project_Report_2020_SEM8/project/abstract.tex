\begin{center}
\thispagestyle{empty}
\vspace{2cm}
\LARGE{\textbf{Abstract}}\\[1.0cm]
\end{center}
\thispagestyle{empty}


\large{\paragraph{}

An abstract is a brief summary of the most important points in a scientific paper/report. Abstracts enable professionals to stay current with the huge volume of scientific literature. Students have misconceptions about the nature of abstracts that may be described as the "table of contents" or "introduction" syndromes. There are several ways to tell if you've written an abstract or not. 

}



\large{\paragraph{}

 An abstract is a brief synopsis or summary of the most important points that the author makes in the paper/report. It is a highly condensed version of the paper/report itself. After reading the abstract, the reader knows the main points that the authors have to make. The reader can then evaluate the significance of the paper and then decide whether or not she or he wishes to read the full paper/report. If one elects to read the full paper/report, further detail is given about each of the significant topics, but no new topics of importance are introduced. If one decides not to read the paper, that decision is based on a knowledge of the paper’s content.Although the abstract appears first in a paper/report, it is generally the last part written. Only after the paper has been completed can the authors decide what should be in the abstract and what parts are supporting detail. }\\


% 5keywords -  2 general and 3 project related
\textbf{Keywords: }Electrocardiogram, DeepLearning, 


\begin{table}[]
\begin{tabular}{|l|l|}
\hline
\textbf{Roll.No} & \textbf{Name} \\ \hline
39               & Aditya        \\ \hline
20               & Mangesh       \\ \hline
39               & Aditya        \\ \hline
20               & Mangesh       \\ \hline
\end{tabular}
\end{table}

