
\chapter{Review of Literature }
\section{Summary of the investigation in the published papers}

Briefly explain the summary of each IEEE / ACM paper or any other literature you are using as part of investigation in your project. 

\section{Comparison between the tools / methods / algorithms}

When submerged in liquids like water, dry ice, which is frequently utilized in imaging, industry, and education, creates a striking fog.  The complex series of events—from immersion to sublimation, bubble ascent, and fog overflow—remains largely unexplored in computer graphics, even though it is frequently shown in videos that highlight considerable fog.  Uncertain production processes make it difficult to replicate this procedure authentically in 3DCG.  Using flames from a combustion simulation and a fog simulation that takes temperature and divergence fields into account, our work presents a visual simulation approach for dry ice sublimation \cite{brookhart2013create}.

\section{Algorithm(s) with example (if applicable)}

Give the pseudo code / algorithm along with explanation. Analysis of algorithm on the basis of parameters like time complexity , space complexity, etc are expected. 